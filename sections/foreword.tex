% SPDX-FileCopyrightText: 2024 Aminda Suomalainen <suomalainen@aminda.eu>
%
% SPDX-License-Identifier: CC-BY-4.0

\documentclass[../wifi-security.tex]{subfiles}
\begin{document}

\section{Foreword}

I have said it multiple times during planning and whoever listens to me, but since this begins the final document, I am going to repeat myself and say that writing about Wi-Fi is difficult.

Wireless networks are everywhere, they are impossible to avoid entirely, even if they were only signs on a public wall saying that wherever you are provides one. Everyone who has used a ''smart'' device has used one and knows what it is, but do they really?

This document attempts to be what I have wanted to read about Wi-Fi for a decade or longer within one cover, but the subject is so vast that I don't know whether I can collect it all or whether I end up covering nearly enough of what I really want to say.

Hopefully you will also begin questioning your WiFi network and its configuration and leave it in a better state than you found it in.

Oh and I am also aiming to graduate by showing that I have some sort of an idea on what is a cybersecurity and how to maintain it.

Later in the document I will also discuss WiFi positioning services, which make the subject even more challenging than it already was. So few people know and understand that the SSID (simplifiedly network name) can easily be turned into a position where the access point is located and the MAC address is even more identifiable that way.

While I attempt to \xblackout{censor} information exposing other people, I am leaving mine visible so there is something to show and there is no guarantee that it's not translatable to my location. Thus I request that you will not try whether or not \texttt{\_nomap} protects me since there is always some service not respecting it and telling you all about my access points regardless.

\end{document}

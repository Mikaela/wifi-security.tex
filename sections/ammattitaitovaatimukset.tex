% SPDX-FileCopyrightText: 2024 Aminda Suomalainen <suomalainen@aminda.eu>
%
% SPDX-License-Identifier: CC-BY-4.0

\documentclass[../wifi-security.tex]{subfiles}
\begin{document}

\selectlanguage{finnish}
\chapter{Ammattitaitovaatimukset}

\section{Kyberturvallisuuden ylläpitäminen}

\begin{quote}
Opiskelija käyttää kyberuhkien hallinta- ja suojautumiskeinoja
\begin{itemize}
\item Suojaa laitteen päivityksillä ja ohjelmistoilla
\item Hallitsee laitetta hallintatyökaluilla
\item Vertailee eri salausmenetelmiä ja valitsee tarkoituksenmukaisen salausmenetelmän
\end{itemize}
Opiskelija hallitsee kyberturvariskejä
\begin{itemize}
\item Valvoa tietoverkkoa hyödyntämällä erilaisia analysointityökaluja
\item Skannata haavoittuvuuksia tarkastelun kohteena olevasta sovitusta verkosta
\item Varmentaa järjestelmien haavoittuvuuksia
\item Tekee kehittämisehdotuksia kyberturvan parantamiseksi
\end{itemize}
Opiskelija edistää kyberturvallisuusratkaisuja
\begin{itemize}
\item Tuntee tietoturvaan- ja tietosuojaan liittyvät lait, asetukset sekä muut viranomaismääräykset
\item Havainnollistaa kyberuhkia ja niitä vastaavia riskejä
\item Noudattaa työtehtävissään tietoturvaohjeita
\item Opastaa kyberturva- tai tietosuoja-asioissa
\end{itemize}
\end{quote}

\section{Järjestelmätuessa toimiminen}

\begin{quote}
Opiskelija toimii järjestelmätuen tehtävissä
\begin{itemize}
\item Viestii järjestelmätuen käsitteillä
\item Seuraa alan kehittymistä ja päivittää omaa osaamistaan
\end{itemize}
Opiskelija hallinnoi järjestelmää
\begin{itemize}
\item Konfiguroi ja hallitsee järjestelmää
\item Käyttää järjestelmän hallintatyökaluja
\item Hallitsee järjestelmän poikkeustilat ja niistä selviytymisen
\item Skriptaa ja automatisoi komentoja
\end{itemize}
Opiskelija tekee järjestelmätason muutoksia
\begin{itemize}
\item Suunnittelee ja valmistelee järjestelmätason muutoksen
\item Aikatauluttaa ja toteuttaa järjestelmätason muutoksen
\item Testaa järjestelmän toiminnan ja dokumentoi toteutetun muutoksen
\end{itemize}
Opiskelija automatisoi päätelaitteen asennuksen
\begin{itemize}
\item Valitsee sopivan sarja-asennusmenetelmän
\item Valmistelee automatisoidun asennuksen
\item Tekee asennuksia päätelaitteisiin valmistelemallaan tavalla
\end{itemize}
\end{quote}

\section{Teknisessä tukipalvelussa toimiminen}

\begin{quote}
Opiskelija tuntee organisaation toimintakulttuurin
\begin{itemize}
\item Toimii sovittujen toimintatapojen mukaisesti
\item Viestii tekniset asiat asiakaslähtöisesti
\item Toimii ja dokumentoi työnsä organisaation tukipalveluprosessien mukaisesti
\item Kuvaa palvelutasosopimuksen vaikutuksen omaan työhön
Opiskelija hallitsee organisaation tietoteknisen ympäristön
\item Tukee työpaikan sovellusohjelmistojen ja käyttöjärjestelmien käytössä
\item Selvittää ja ratkaista laitteisto-, ajuri-, verkko- ja tulostusongelmia
\item Toimii järjestelmätasolla halliten käyttäjätilejä
\end{itemize}
Opiskelija ratkaisee asiakkaiden palvelupyyntöjä
\begin{itemize}
\item Käsittelee ja luokittelee sekä uudelleenohjaa tukipyyntöjä
\item Tukee ja opastaa asiakasta sekä hallitsee laitetta tarvittaessa etäyhteydellä
\item Ratkaisee asiakkaiden palvelupyyntöjä
\end{itemize}
\end{quote}

%\section{Perustehtävät}

%\begin{quote}

%\end{quote}

\selectlanguage{english}

\end{document}

% SPDX-FileCopyrightText: 2024 Aminda Suomalainen <suomalainen@aminda.eu>
%
% SPDX-License-Identifier: CC-BY-4.0

\documentclass[../wifi-security.tex]{subfiles}
\begin{document}

\chapter{Taking action}

If everything older than WPA3 is broken and we make ourselves vulnerable through backwards compatibility, perhaps we should run guest networks usable by everyone as the precense of open network is easier for anyone to confirm than someone breaking into the network through captured four way handshake.

\section{Open Wireless Networks}

Also known as unprotected or insecure networks although that may be inaccurate with WPA3 also bringing Open Wireless Enhanced which provides encryption between the access point and client, while downgrade attacks and evil twin attacks still exist.

That may sound scary, but \dots

\subsection{Combating climate change}

% TODO GREEN CODE citation or maybe even a direct quote?

When considering the carbon footprint of Information Technology and the internet in general, researchers end up to conclusions that include mobile networks being more energy-intensive than static ones. Thus having an open guest network transfer data instead of connecting 4G BBS may be better for climate.

Our WiFi access point is going to consume electricity whether or not we are using it or even present. If it was open, maybe this would help reduce our carbon footprint.

Reducing carbon footprint and improving cybersecurity both work in layers, small steps cumulate and become more impactful when put together.

\subsection{Helping those who just need access}

Have you ever needed access to a WiFi network in the middle of habitation with a lot of closed networks available, without any open ones? What is the easiest thing to do in this case?

Opening Router Keygen by YoloSec, selecting a SSID with default configuration and seeing possible passwords to it, problem solved.

Of course that isn't legal (unless the access point belongs to your confused partner who tells you to show them and resulting to reconfiguration of said access point for security reasons), but will a blackhat who just needs internet access care about that?

Encountering a blackhat in such a situation may be bad luck, but if there was an open network, they would have less incentive for attacking us or neighbouring networks and hopefully they might pay it forwards by opening their network too.

% TODO! Cite https://github.com/yolosec/routerkeygenAndroid

\subsection{Reducing Radio Frequency pollution}

This paper repeats it a lot, but many networks come with default configuration that uses maximum transmit power, maximum channel width and crowded channels. Would open networks help with that, especially with less radios having been active for communicating with mobile BBS? Maybe it would at least reduce the amount of Wi-Fi tethering.

\section{Staying safe}

I believe the zero trust approach can be adapted here. If we treat every network as possibly compromised, as a result we will take actions against the router and potential attacker and thus it doesn't matter where our devices are.

\begin{itemize}
\item Encrypt everything. DNS encryption is supported by current versions of all major operating systems and can be enforced system wide e.g. through group policy and similarly https-only mode can be enabled. % TODO BLOGI! Ja https://www.eff.org/deeplinks/2020/01/why-public-wi-fi-lot-safer-you-think !
\item Firewall everything and consider carefully what to let through the firewall.
\item Reject passwords, embrace stronger authentication methods. Especially in remote control, such as SSH, the best practices include \texttt{PasswordAuthentication no}, and using keys instead. Thus even if SSH is exposed in firewall, it's hardened against casual bruteforce attackers.
\end{itemize}


\end{document}

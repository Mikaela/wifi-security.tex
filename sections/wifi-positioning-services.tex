% SPDX-FileCopyrightText: 2024 Aminda Suomalainen <suomalainen@aminda.eu>
%
% SPDX-License-Identifier: CC-BY-4.0

\documentclass[../wifi-security.tex]{subfiles}
\begin{document}

\chapter{WiFi positioning services}

\section{WiFi Privacy isn't only between your client and AP}

Since the advent of GPS personal navigation technology has been becoming increasingly common and in addition to cars, smartphones and even smartwatches or smart rings may come with a GPS nowadays.

However the traditional positioning systems may not function well when the client is located between high buildings, indoors or even tunnels and at some point, businesses decided to start collecting WiFi Access Point MAC addresses and locations for reliable positioning anywhere with a WiFi access.

The largest of these databases likely belong to Google and Apple
%and Microsoft
considering how with default settings their operating systems Android and iOS contribute into them.

%\subsection{TODO: Microsoft}

%TODO: How does Microsoft do it? Any Microsoft app with access to local networks? They do it \autocite{Microsoft_nomap}, but is it just any app that is given access to location and nearby devices?

\section{Google}

When an Android phone is first booted, there is an setup wizard that amongst Google account credentials asks whether to enable Google's Location Services. Later in settings there is also a toggle on whether WiFi network scanning is allowed even when WiFi is disabled.

Should the user answer yes, which they likely will, all networks the phone sees are transmitted to Google both for locating the device now, and in the future.

%\subsection{Street View Controversy and \_nomap}

%In addition to Android, Google also collected WiFi network positions through the Street View Cars in

%\subsection{Mozilla Location Services}

%\subsection{BeaconDB}

%\subsection{Positon}

%\section{Apple}

%\section{Wardriving}

%\subsection{WiGLE}

\printbibliography[title={Chapter references}]

\end{document}

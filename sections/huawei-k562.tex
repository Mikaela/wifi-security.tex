% SPDX-FileCopyrightText: 2024 Aminda Suomalainen <suomalainen@aminda.eu>
%
% SPDX-License-Identifier: CC-BY-4.0

\documentclass[../wifi-security.tex]{subfiles}
\begin{document}

\section{Introductionary case study: Huawei K562}

Also known as \textit{OptiXstar K562 Ethernet Terminal}, \textit{DNA Mesh WiFi -modem K562} and \textit{Valoo WiFi 6 Mesh -router}. First we factory reset it, then we will go through all the settings to see how everything is wrong.

\subsection{Fresh from the factory}

I was trying to perform the setup using an ethernet cable, but since that is very slow for some reason, I am laxing from information security a bit by connecting to the device wirelessly.

% TODO: MAYBE PHOTOS HERE?

The default credentials read on the bottom of the device and luckily either Huawei or DNA has included a QR code there, so first we use an Android to open network settings and scan the code. Next we select the new network and share the connection, so we can see the password without having to take photos or think of other hacks.

The control panel is located in \texttt{192.168.101.1} with the username being \texttt{admin} and password \texttt{1234}.

\fbox{\includegraphics[height=11.5cm]{huawei/00-login.png}}

\subsubsection{Admin password}

Now if security was a priority, we would be forced to change the login password immediately. However as they don't, we have to navigate More, System Management, Account Management by ourselves to see this prompt.

\fbox{\includegraphics[height=11.5cm]{huawei/03-accountmanager.png}}

We are thrown back to the login screen displayed previously.

\fbox{\includegraphics[height=11.5cm]{huawei/00-login.png}}

\subsubsection{Default SSID}

For some reason this device shíps with one 2.4 GHz SSID enabled, DNA-WIFI-\Name, and two 5 GHz ones, DNA-WIFI-\Name and DNA-WIFI-5Ghz-\Name. As the defaults are insecure, we change them immediately (which will also permit my devices to reconnect after the factory reset).

We have to navigate to WLAN and there both 2.4 GHz and 5 GHz settings separately.

We also have to disable WPS by ourselves by disabling the checkbox (on both frequencies).

% TODO WPS insecure

Next the same for WiFi 5, where note the existence of a duplicate SSID!

Off-screen I have also visited Network Configuration, LAN Settings, which is irrelevant for this work, but my devices have static network configuration that is not compatible with Huihai defaults.

\subsection{My WiFi}

Now that our basics are in order, we are ready to start moving through all the menus. Homepage and Internet Access have nothing terribly interesting, but our first issue is My WiFi where we can see a \textit{WiFi Power Mode}, which is set to \textit{Through-wall (high power, better signal)}.

Now, is this a reasonable option? I am living in a studio apartment and I have a single room (the kitchen closet has its removed) and I don't use WiFi or anything else in the bathroom.

Remember that WiFi is a semi-duplex radio, so all access points sharing the same channel distrupt each other due to sharing airtime assuming they can hear each other. We are also turning all radios connecting to the access point to maximum power since there is an assumption that if server with maximum power is heard, so will be the client with equal power, so there may also be a question of electricity and green ICT perspective. On the security side, do we want the network to be connectable from far beyond the closest set of walls? \autocite{metistxpower}

We change it to \textit{Sleep (low power, weak signal)} and receive a scary warning \textbf{You are advised to use the through-wall mode. The signal in this mode is better.Wi-Fi power modification takes effect, causing a short interruption of Wi-Fi.}

% At this point our \texttt{microsoft-edge-stable} freezes and crashes, so we have to restart it before continuing to...

% TODO: what is WMM here?

\subsection{WLAN Advanced Configuration}

Here we see that our device is configured to use

Channel: automatic
Channel width: automatic 20/40 MHz
Mode: 802.11b/g/n/ax
Airtime fairness

Channel: automatic
Channel width: automatic 20/40/80/160 MHz
Mode: 802.11a/n/ac/ax
Airtime fairness

The device again repeats \textbf{You are advised to set it to 11b/g/n/ax.}

\subsubsection{2.4G Wi-Fi all wrong}

2,4 G frequencies are a limited resource, one that our access points share with microwave ovens, bluetooth and everything. If we want to live in peace without disturbing others, we have three channels, that is if everyone uses 20 MHz channel width, but no, a lot of devices decide to use 40 MHz.

As per Apple's recommendation we switch to channel width 20 MHz.\autocite{appleap}

Now, the mode. While generally backwards compatibility is good, we are using the valuable shared airtime and by being backwards compatible, we make sacrifices including in speed (and possibly energy consumption). \autocite{doubleedgedbackwards}

802.11n, nowadays also known as Wi-Fi 4 was adopted in 2009 and remains the most common version of 2.4G Wi-Fi today. 802.11g (Wi-Fi 3) is from 2003 and 802.11b (Wi-Fi 1) from 1999 \autocite{wikipediawifi6}. Do we really need backwards compatibility so far back?

The only Wi-Fi device incapable of WiFi N that I can name is Nokia E63 running Symbian. Thus we switch to 802.11n-only.
% TODO: the sentence continued like below, but that is not factual since it has been touched by more recent standards and likely will.
% which is the most modern and likely last 2.4 GHz option.

\subsubsection{Interlude: WiFi DFS channels and Location Aware Routing}

The automatic channel selection of 5 GHz sounds great in theory, but in practice we are following European regulation and thus can only use channels 36 to 64 if we cannot detect weather radars (100-144) \autocite{metisdfs} and who knows what are channels 149-177 for (TODO).

Our devices should be aware of what country we are in, but commonly consumer devices don't allow configuring that explicitly so we end up in Germany if we are even in Europe. This can be seen through \texttt{iw reg get} for our current NIC when using Linux:

\begin{verbatim}
global
country FI: DFS-ETSI
...

phy#0 (self-managed)
country FI: DFS-UNSET
...
\end{verbatim}

Redacted are the regulations our WiFi network interface card recognises. Luckily at the time of writing, we are correctly detected to be in Finland, but that is often not the case, and generally all access points broadcast being in Germany.

However note that DFS is unset, thus we are avoiding DFS channels out of the box and have to configure our device manually to have free channel.

DFS stands for Dynamic Frequency Selection and means that when booting, our access point scans around for weather radars for 10 minutes before beginning broadcasting and WiFi starting to work.

We can also see what the nearby access points broadcast:\\ \texttt{sudo iw wlan0 scan | grep -E "SSID:|Country:" | less}

\begin{verbatim}
	SSID: LocationServices_nomap
	Country: FI     Environment: bogus
	SSID: LocationServices_nomap
	Country: FI     Environment: bogus
\end{verbatim}

All of the more presumably recent DNA devices appear to be advertising being setup in Finland, but in invalid environment, while the older models advertise being setup in Germany Indoor/Outdoor meaning the access points don't know are they setup indoors or outdoors, which again furter restricts the frequency options.

\subsubsection{Back to 5G configuration}

In order to not share channel with all 14 other networks that are present at the time of writing, we must configure the 5G channel by hand. As the regulatory domain says DFS is unset, everyone keeps avoiding those channels and we should check where the weather radars are. I mostly spend my time in Helsinki and Kotka and happen to know that the closest weather radars to those share the channel 128 and are located in Vantaa and Anjalankoski \autocite{metisweather}. Thus we are free to pick any other channel.

I have once observed a TP-Link device on channel 100 at my home, so I am using the channel 112 which will not overlap the 128 which would cause a distruption whenever the weather radar activates.

Thus we again see Huwaei telling us \textbf{You are advised to set the channel to Automatic.}, since they seem to be having incorrect assumption of access point being configured properly. Perhaps they expect to be a minority and rely on the majority to have proper configuration for Location Address Networking.

When applying this, Huawei once again gives us a warning \textbf{Due to frequency regulation, if you select this channel, you need to wait for 10 minutes until the Wi-Fi network is available.}, but that we already knew, even if otherwise it might have been helpful.

Next would be \textit{Wi-Fi Coverage Management}, which doesn't refer to the transmit power discussed previously, but mesh networking and I am going to just disable this due to the small size of my apartment.

Automatic Wi-Fi Shutdown is also irrelevant and as I only have this Huawei router I don't need to switch from router to bridge mode either.

\subsection{LAN Settings}

I visited here before and the Apple recommendation is 8 hours for private networks, one for public hotspots and guest networks.\autocite{appleap}

We can no longer reliably identify devices based on the MAC address and that affects routers as well, since modern mobile devices have MAC address randomisation with unpredictable interval. In case of Android, there is a developer mode option to always change MAC address.

\subsection{IPv6 and DHCPv6 Server Configuration}

IPv6 itself is a toggle whether it's enabled or not, and we are going to leave it enabled, since most of Finland already has IPv6 available and IPv4 addresses ran out ages ago.

In DHCPv6 settings, we could change DNS Information to benefit from DNS Over TLS opportunistic mode. % TODO

% Quad9 (Secure)  https://dns.quad9.net/dns-query dns.quad9.net   2620:fe::fe     2620:fe::9      9.9.9.9 149.112.112.112 https://docs.quad9.net/Setup_Guides/iOS/iOS_14_and_later_%28Encrypted%29/#download-profile       no      https://www.quad9.net/support/faq/#edns

I choose \texttt{2620:fe::fe} and \texttt{2620:fe::9} which are Quad9 with ECS and increase client security by DNS level filtering of bad domains. More on ECS later. % TODO

We could also enable ULA in automatic mode. ULA starts for Unique Local Address and is the successor of the old class A/B/C internal networks from IPv4.

The final option of \texttt{Network Configuration} would be \texttt{UPnP} which for security should be disabled since it could allow potentially unwanted applications to connect to the internet through our router firewall, but I run firewalls on all client devices where possible and utilise applications relying on UPnP such as Syncthing synchronising my files between my devices without other people's computers, so I am leaving it enabled.

\subsection{Security configuration}

The router admin panel would begin at MAC address filtering, but as mentioned before, mobile devices are constantly changing their MAC addresses and besides as discussed in the upcoming Aircrack subsection, the MAC addresses or BSSIDs are visible really easily and thus MAC address filtering is entirely pointless. Otherwise there isn't really much to say about security configuration and unlike everything before, the defaults seem reasonable.

The next subsection would be \textit{System Management}, where the only relevant thing would be \textit{Account Management}, which was already handled.

\printbibliography[title={Chapter references}]

\end{document}

\documentclass[../wifi-security.tex]{subfiles}
\begin{document}

\chapter{WiFi Monitoring mode}


As previously discussed, % TODO: time travel is fun
the first step is switching the NIC to monitoring mode so we can observe the WiFi traffic around us.
In case of Tails, this means three simple steps.

\begin{enumerate}
	\item Open the \texttt{root terminal} application
	\item Stop wpa\_supplicant so it won't interfere by running \texttt{systemctl stop wpa\_supplicant}. Note that this will cut our wireless connection.
	\item Actually switch to the monitoring mode by \texttt{airmon-ng start wlan0}
\end{enumerate}

%Datapaja,FInlandia_public. homerun1x

\section{aircrack-ng}

Still staying in the root terminal, we execute \texttt{airodump-ng wlan0} and we should now see all the networks around us alongside client devices (stations) requesting connection to BSSIDs and immediately learn that security is a lie.

Based on short observation, I know someone around me has visited both \texttt{Finlandia\_public} and \texttt{Datapaja}. I also know that a neighbor named \Name who likely lives in \texttt{Koti\_\Name} and they own an iPhone 13 Pro Max.

Thus we have confirmed that hidden SSIDs (``\texttt{<length: 0>}'') are harmful for your privacy and won't stop us from knowing them anyway.

\section{Kismet}

See also wardriving in wifi-position-services?

\section{WireShark?}

\end{document}
